%Formatting Guidelines for Writing Dissertation.
\chapter{Introduction}\label{Guidelines}
\chapter{Introduction}
\section{Introduction}
Misinformation spreads quickly on social media which often uses both text and images, that makes detection a multimodal challenge. Traditional approaches usually handle text and images separately or combine them in a basic, straightforward manner. This weak integration results in poor semantic alignment between modalities that struggles in cases where text and image contradict each other, and offers little transparency into why a prediction was made. Models such as SpotFake use BERT for text and VGG19 for images with equal-weight feature fusion, but this approach offers limited interaction between the modalities and often faces trouble to generalize to complex, real-world misinformation.

Recent advances in multimodal learning provide more effective solutions. First, \textbf{contrastive learning} (in a CLIP-style setup) aligns text and image embeddings by bringing related pairs closer and pushing unrelated pairs apart thus improving cross-modal understanding before supervised training. Second, \textbf{cross-modal attention} allows the model to selectively focus on the most informative modality or token for each post. Finally, \textbf{explainability techniques} such as Grad-CAM for images and token-level SHAP for text helps in finding the reasoning behind predictions, enhancing transparency and supporting more thorough error analysis. Additionally, replacing VGG19 with ResNet50 strengthens visual encoding by offering better representations and inductive biases.

This project integrates these advancements into a single, end-to-end pipeline for fake news detection on Twitter posts. The system employs BERT for textual encoding, ResNet50 for image encoding (with comparative analysis against VGG19), contrastive pre-training for cross-modal alignment, cross-modal attention for intelligent fusion, and explainability techniques to make the decision-making process interpretable.

\section{Objectives}
The primary objectives of this project are:
\begin{itemize}
    \item To design a multimodal architecture combining BERT and ResNet50 (and compare against VGG19) for robust text–image encoding.
    \item To pre-train the text and image encoders using CLIP-style contrastive learning thus improving the alignment between modalities and enhancing overall performance on downstream tasks.
    \item To introduce cross-modal attention for observation-wise and selective fusion instead of simple concatenation.
    \item To integrate explainability methods such as Grad-CAM (for image regions), SHAP (for token-level importance), and attention heatmaps (for text–image focus) to make predictions transparent.
\end{itemize}

\section{Organization of Project}
This project is structured into six chapters and each chapter is built based on the previous to present a complete picture of the research process:

\begin{itemize}
    \item \textbf{Chapter 1} introduces the problem of multimodal fake news detection, presents the motivation for this research, and defines the project objectives.
    \item \textbf{Chapter 2} reviews the related work on multimodal misinformation detection, contrastive pre-training, attention-based fusion, and explainability in vision–language models.
    \item \textbf{Chapter 3} details the proposed methodology that includes encoder selection (VGG19 vs ResNet50), CLIP-style contrastive pre-training, and cross-modal attention-based fusion.
    \item \textbf{Chapter 4} presents the experimental results, cross-modal alignment analyses, and explainability visualizations (Grad-CAM, SHAP, and attention maps).
    \item \textbf{Chapter 5} discusses the limitations of the current approach, ethical considerations, and potential future work, such as incorporating temporal signals, enabling few-shot learning, and enhancing robustness to distribution shifts.
    \item \textbf{Chapter 6} concludes the project with key findings and their practical implications for improving fake news detection systems.
\end{itemize}


\chapter{Literature Review}
Probabilistic forecasting has emerged as an essential tool for managing uncertainty across multiple domains—electricity price forecasting, wind power forecasting, stock price prediction, and web traffic forecasting. Classic point forecasting techniques often do not and cannot, by their very nature, fully capture the intrinsic variability at play in these fields, making probabilistic approaches more appealing to risk-sensitive decision-making. In the area of electricity price forecasting, models that deliver probability distributions of prospective prices have become increasingly popular. Two popular probabilistic forecasting methods, namely the LUBE (Lower Upper Bound Estimation) method and QR (Quantile Regression) based method, are proposed by \cite{12} and \cite{13}, which have given rise to various other methods that have incorporated these two methods. A combined probabilistic forecasting system, CPFS, synergizing quantile regression with neural networks, has established its superiority over singular model methodologies in both interval width and coverage probability and, therefore, given more reliable forecasts \cite{9}. Similarly, Gaussian Processes have been used in wind power forecasting to help in modeling the interaction that might exist between wind power generation and meteorological conditions. \cite{10} presents a multitask Gaussian process (MTGP) method successfully applied to scenarios with scarce historical data by transferring knowledge from similar source tasks to enhance the predictions of new wind farms.

In the case of prediction for stock prices, deep learning architecture-based models such as LSTM, GRU, and CNN can effectively capture the complexity and non-linearity observed within financial markets. Other models are uncertain, and there is a need to use probabilistic methods such as QR and KDE to increase the reliability of the forecast \cite{4, 9}. Deep learning architectures, particularly Long Short-Term Memory (LSTM) networks and Gated Recurrent Units (GRU), have exhibited the ability to encompass the subtle complexities inherent in the data; however, their effectiveness can be enhanced through the use of non-parametric approaches, providing better coverage intervals and better decision-making capabilities \cite{8, 3}. The application area also includes web traffic prediction, sharing patterns with similar spiky erratic changes in stock price forecasting, which also benefits from the combination of machine learning techniques. Methods such as SVM, with the incorporation of ensemble methods, have been employed to predict web traffic, thereby allowing for the allocation of resources and strategic planning of websites within precise prediction intervals \cite{5, 7}. In addition, swarm intelligence methods, including PSO and WOA, have successfully been implemented for hybrid models in forecasting; this has improved the precision of predictions in dynamic conditions \cite{9}. The key development in this case is the implementation of hybrid models integrating probabilistic forecasting with machine learning and optimization techniques, making the establishment of higher accuracy of forecasting within these fields more achievable and providing greater resilience to data uncertainty \cite{1, 6}. In conclusion, probabilistic forecasting remains fundamental in the fight against challenges attached to volatility in industries such as energy, finance, and online services, wherein uncertainty and nonlinearity are integral to the decision-making process \cite{2, 10, 9}.

Recent advancements further underscore the potential of hybrid models. For instance, \cite{14} discuss blending climate predictions with AI models to enhance hydroclimatic variable predictability, emphasizing the benefits of combining dynamical and data-driven models. Similarly, \cite{15} propose two-stage hybrid models to address heterogeneity in time series data, demonstrating improved forecasting performance by capturing both global and local patterns. In the realm of wind power forecasting, \cite{16} introduces an adaptive quantile regression approach resilient to missing values, highlighting the importance of robustness in probabilistic models. Additionally, \cite{17} present a probabilistic forecasting method for regional solar power that incorporates weather pattern diversity, showcasing the adaptability of hybrid models to various energy domains.

Despite these advancements, several research gaps remain. Most existing methods struggle to simultaneously ensure high prediction interval coverage probability (PICP) and maintain narrow prediction intervals (PINAW). For instance, QR-based methods often produce precise but under-confident forecasts, while LUBE-based methods may offer good coverage but at the cost of excessively wide intervals. There is also a lack of standardized benchmarking across different confidence levels and datasets, making it difficult to generalize findings. Furthermore, many methods are either purely parametric or purely data-driven, limiting their robustness in real-world applications where noise and variance are high.

These limitations provide a strong motivation for the development of novel hybrid methods. By combining the strengths of LUBE-based interval estimation with statistical or deep learning-based enhancements, it is possible to strike a better balance between reliability and efficiency. For instance, hybridizing LUBE with Quantile Regression (QR) introduces complementary strengths like LUBE's high coverage with QR’s narrower interval width. Similarly, integrating LUBE with Parametric models such as the Weibull distribution enables modeling of residual uncertainty and tail behavior more effectively, improving coverage in volatile scenarios. These hybrid approaches are designed to adapt across multiple model architectures (e.g., LSTM, GRU, CNN, BiLSTM) and datasets (e.g., stock prices, electricity load, web traffic), ensuring more generalizable and practically applicable forecasting systems. Thus, the present work addresses these gaps by proposing and evaluating such hybrid probabilistic forecasting frameworks that offer robustness, adaptability, and improved decision-support capacity in uncertain environments.

\begin{table}[!ht]
    \caption{Recent Literature on Probabilistic Time Series Forecasting Methods.} \vspace{0.2cm}
    \centering
    \resizebox{\textwidth}{!}{ 
    \begin{tabular}{|p{0.5cm}|p{2.8cm}|p{2cm}|p{2cm}|p{3.2cm}|p{3.2cm}|p{2.5cm}|p{2.5cm}|}
        \hline
        \textbf{Id} & \textbf{Reference} & \textbf{Model} & \textbf{Method} & \textbf{Merits} & \textbf{Drawbacks} & \textbf{Dataset} & \textbf{Accuracy Measure} \\ \hline
        \cite{12} & Abbas Khosravi et al. (2011), IEEE transactions on Neural Networks & Neural Networks & LUBE (Lower-Upper Bound Estimation) & Produces reliable prediction intervals without assuming noise distribution. Applicable to various forecasting problems. & Tuning of custom loss functions is complex. Training instability may occur. & Benchmark time series datasets & PICP, PINAW \\ \hline
        \cite{2} & Cameron Cornell et al. (2024), Int. Journal of Forecasting & Neural Networks & Probabilistic Forecasting for Electricity Prices & Adapts well to market volatility, integrates market mechanisms in forecasting. & Model sensitivity to sudden price spikes and data availability issues. & Australian National Electricity Market (NEM) & CRPS, RMSE, PICP \\ \hline
        \cite{9} & Yan Xu et al. (2024), Computers and Industrial Engineering & Quantile Regression & Quantile Combination Forecasting & Combines multiple quantile forecasts to improve robustness and accuracy. & Complexity in selecting and weighting contributing models. & Electricity price datasets & PICP, Quantile Loss \\ \hline
        \cite{1} & Sourav Kumar Purohit and S. Panigrahi (2024), Information Sciences & Hybrid (Statistical + DL) & Optimized Deep Learning & Integrates classical and deep models for improved oil price forecasting. & High computational demand; interpretability challenges. & Crude oil market datasets & RMSE, MAE, PICP \\ \hline
        \cite{3} & Hao-Hsuan Huang and Yun-Hsun Huang (2024), Energy Reports & Deep Learning & Probabilistic Forecasting with Weather Diversity & Accounts for weather diversity in regional solar forecasting. Enhances generalization. & Relies heavily on accurate meteorological inputs. & Regional solar power datasets & PICP, PINAW, RMSE \\ \hline
    \end{tabular}
    }
    \label{tab:consolidated_lit_review}
\end{table}


Table ~\ref{tab:consolidated_lit_review} summarizes recent literature on probabilistic time series forecasting methods. It highlights key studies involving different models and approaches, such as Neural Networks with LUBE, Quantile Regression, Hybrid statistical-deep learning methods and deep learning techniques tailored for specific applications like electricity price and solar power forecasting. For each reference, the table outlines the main merits and drawbacks, datasets used, and accuracy measures employed. This consolidated view provides insight into the strengths and limitations of various probabilistic forecasting methods across diverse domains, guiding future research directions.
